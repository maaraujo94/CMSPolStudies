\documentclass{article}

\usepackage{graphicx}

\usepackage{geometry}
\usepackage{amsmath}
\usepackage{indentfirst}
\usepackage{hyperref}
\usepackage{multirow}
\usepackage{comment}
\usepackage{braket}
\usepackage{xcolor}

\newcommand{\pt}{p_\text{T}}
\newcommand{\cost}{\cos\theta}

\newcommand{\loc}{/home/mariana/Documents/2020_PhD_work/CERN/CMSPolStudies/Jpsi/2018/bkgFits}

\begin{document}

\title{$J/\psi$ polarization workflow }
\author{Mariana Ara\'ujo (LIP)}
\maketitle

\tableofcontents

\pagebreak

\section{Cheat sheet: code order}

\begin{enumerate}
\item histoSave.C
\item getCos.C
\item bkgSave.C
\item MCmass\_[...].C (optional - illustrative plots)
\item ltBkg.C
\item plotLtPars.C
\item mBkg.C
\item plotDMPars.C
\item bkgCosth.C
\item fitBkgCosth.C / fitBkgCosth2d.C
\item (only for 2d fit) plotCosPars.C
\end{enumerate}

\pagebreak

\section{PR SR $\pt$ vs $\cost$}
\subsection{Cuts and binning} 

Cuts on stored datafiles:
\begin{itemize}
\item $M(\mu\mu)\in[3.0,3.2]$ GeV
\item $|y(\mu\mu)|<1.2$
\item $|\ell|<0.01$ cm
\end{itemize}

Fine binning for data and MC
\begin{itemize}
\item $25-37$ GeV: 12 bins of 1 GeV
\item $37-46$ GeV: 6 bins of 1.5 GeV
\item $46-76$ GeV: 12 bins of 2.5 GeV
\item $76-100$ GeV: 6 bins of 4 GeV
\item $100-120$ GeV: 2 bins of 10 GeV
\end{itemize}

\subsection{Storing histograms}

Run histoSave.C to store the 2d histos ($\pt$ vs $\cost$) in histoStore.root.

\subsection{Limit $\cost$}

Run getCos.C to get the limit $\cost$ in each $\pt$ and interpolate a function to get $\cost_{max}(\pt)$

\pagebreak

\section{Background fitting: Mass and Lifetime distributions}
\subsection{Regions and binning} 

The three mass regions and the two lifetime regions are defined according to fixed limits
\begin{equation}\begin{cases}
m^{LSB}&\in[2.92, 2.95] \text{ GeV}\\
m^{SR}&\in[3.0, 3.2]  \text{ GeV}\\
m^{RSB}&\in[3.21, 3.28]  \text{ GeV}\\
\end{cases}\hspace{0.5cm}\text{and}\hspace{0.5cm}
\begin{cases}
|\ell^{PR}| < 0.01\text{ cm} \\
0.014 < \ell^{NP} < 0.05\text{ cm}
\end{cases}
\end{equation}

Coarse binning for background fitting
\begin{itemize}
\item $25-46$ GeV: 3 bins of 7 GeV
\item $46-76$ GeV: 3 bins of 10 GeV
\item $76-100$ GeV: 1 bin of 24 GeV
\end{itemize}

\subsection{Storing histograms}

Run bkgSave.C to store the 1d histos (mass in the prompt region and lifetime in the signal region) for all $\pt$ bins. Stores data, MC and data/MC ratio.


\subsection{Lifetime fit}

Fit the $\ell$ region $[-0.01,0.05]$ cm (mass in the signal region). For the PR contribution (equal to the resolution), consider a double Gaussian distribution:
\begin{equation}
L_{PR}=L_{res} = (1-f)\cdot g_{G_1}(\ell|\mu, \sigma_1)+f\cdot g_{G_2}(\ell|\mu, \sigma_2),
\end{equation}

Both Gaussian functions share a common $\mu$ and have independent $\sigma_{1,2}$. For the NP contribution, consider a negative exponential convoluted with the double Gaussian defined above:
\begin{equation}
L_{NP}=	\left[\exp(-\ell/\lambda)\cdot\mathcal{H}(\ell)\right]*L_{res}(\ell|\mu, \sigma_1,\sigma_2),
\end{equation}
where $\mathcal{H}$ is the Heaviside step function.

Running a 2d fit with constant $f=0.14$ and $\mu=0$, free $\sigma_{1,2}$, $\lambda$, and normalizations $N_{PR}$, $N_{NP}$.

\subsubsection{Running the fit}

Run ltBkg.C to run the 2d fit reading the histos from bkgSave.C and outputting all fit parameters as well as $\chi^2/$ndf and NP fraction per $\pt$ bin, $f_{NP}$. Plots distribution, pulls and relative difference.

\subsection{Mass fit}

Fit the mass region $[2.94, 3.26]$ GeV (lifetime in the prompt region). For each $\pt$ bin, fit the signal with a double crystal ball function:
\begin{equation}
f\cdot g_{CB_1}(m_{\mu^+\mu^-})+(1-f)\cdot g_{CB_2}(m_{\mu^+\mu^-}),
\end{equation}
where a crystal ball function $g_{CB}(m_{\mu^+\mu^-})$ is defined by
\begin{equation}
g_{CB}(m) = \begin{cases}
\frac{N_{SR}}{\sqrt{2\pi}\sigma_{CB}}\exp\left(-\frac{(m-\mu)^2}{2\sigma^2_{CB}}\right), & \text{for }\frac{m-\mu}{\sigma_{CB}} > -\alpha \\
\frac{N_{SR}}{\sqrt{2\pi}\sigma_{CB}}\left(\frac{n}{|\alpha|}\right)^n\exp\left(-\frac{|\alpha|^2}{2}\right)\left(\frac{n}{|\alpha|}-|\alpha|-\frac{m-\mu}{\sigma_{CB}}\right)^{-n}, & \text{for }\frac{m-\mu}{\sigma_{CB}}\leq-\alpha
\end{cases}.
\end{equation}
Both CB functions share a common $N_{SR}$, $\mu$, $n$ and $\alpha$, and have independent $\sigma_{1,2}$. The background is modeled with a negative exponential
\begin{equation}
N_{BG} \text{e}^{- m / \lambda}
\end{equation}

Running a 2d fit with constant $f$ and $\mu$, linear $\sigma_{1,2}$ and free $\lambda$ and normalizations $N_{SR}$, $N_{BG}$. The tail parameters $n$ and $\alpha$ are fixed to results of the MC studies (1.2 and 2.15, respectively). 

\subsubsection{Running the fit}

Run ltBkg.C to run the 2d fit reading the histos from bkgSave.C and outputting all fit parameters as well as $\chi^2/$ndf and mass background fraction per $\pt$ bin, $f_{BG}$. Plots distribution, pulls and relative difference.

\subsection{Plotting the fit parameters}

The lifetime fit parameters are plotted with plotLtPars.C. This plots:
\begin{itemize}
\item All 7 fit parameters
\item $f_{NP}$
\item $\sigma=\sqrt{\sigma_1^2+\sigma_2^2}$
\item rms$=\sigma\cdot\pt/M$
\item $\sigma_2/\sigma_2$
\end{itemize}

The mass fit parameters are plotted with plotDMPars.C. This plots:
\begin{itemize}
\item All 9 fit parameters
\item $f_{BG}$
\end{itemize}

\pagebreak

\section{Background fitting: $\cost$ distributions}
\subsection{Background selection}

NP background selection:
\begin{itemize}
\item $M(\mu\mu)\in[3.0,3.2]$ GeV
\item $|y(\mu\mu)|<1.2$
\item $0.014 < \ell < 0.05$ cm
\end{itemize}

LSB background selection:
\begin{itemize}
\item $M(\mu\mu)\in[2.92,2.95]$ GeV
\item $|y(\mu\mu)|<1.2$
\item $|\ell| < 0.01$ cm
\end{itemize}

RSB background selection:
\begin{itemize}
\item $M(\mu\mu)\in[3.21,3.28]$ GeV
\item $|y(\mu\mu)|<1.2$
\item $|\ell| < 0.01$ cm
\end{itemize}

\subsection{Storing histograms}

Run bkgHisto.C to store the 2d histos ($\pt$ vs $\cost$) of data/MC for the three background regions.

\subsection{$\cost$ fit}

For each $\pt$ bin, fit the distribution with
\begin{equation}
f_{\cost} = N\left(1+\lambda_2(\cost)^2+\lambda_4(\cost)^4\right)
\end{equation}

For the NP case, $\lambda_2$ is redefined as $\lambda_{NP}$ and $\lambda_4=0$.

Running a 1d fit with normalization $N$ free and $\lambda_2$, $\lambda_4$ successively fixed to their central values (considered constant in $\pt$).

Also running a 2d fit with normalization $N$ free and $\lambda_2$, $\lambda_4$ constant in $\pt$.

\subsubsection{Running the fit}

Run fitBkgCosth.C to run the 1d fit or fitBkgCosth2d.C to run the 2d fit. Outputs all fit parameters per $\pt$ bin. Plots distributions, pulls and relative difference.

\subsection{Plotting the fit parameters}

The 1d fit automatically plots the results. For the 2d fit, run plotCosPars.C. This plots:
\begin{itemize}
\item $N$ and $\lambda_{NP}$ for NP fit
\item $N$, $\lambda_2$ and $\lambda_4$ for sideband fits
\end{itemize}

\pagebreak

\section{Background subtraction}

\subsection{Getting the background distributions}

Generate the background $\cost$ distributions using the above fit parameters, in the fine binning of the data

\end{document}
