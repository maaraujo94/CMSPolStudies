\documentclass{article}

\usepackage{graphicx}

\usepackage{geometry}
\usepackage{amsmath}
\usepackage{indentfirst}
\usepackage{hyperref}
\usepackage{multirow}
\usepackage{comment}
\usepackage{braket}
\usepackage{xcolor}
\newcommand{\pt}{p_\text{T}}
\newcommand{\cost}{\cos\theta}
\newcommand{\jpsi}{\text{J}/\psi}

\newcommand{\loc}{/home/mariana/Documents/2020_PhD_work/CERN/CMSPolStudies/Jpsi/2018/bkgFits}

\begin{document}

\title{$J/\psi$ polarization workflow }
\author{Mariana Ara\'ujo (LIP)}
\maketitle

\tableofcontents

\pagebreak

\section{Cheat sheet: code order}
\subsection{PR\_fit}

\textbf{Initial storage}
\begin{enumerate}
\item histoSave.C
\item (cosMax) histoSave.C + getCos.C
\item bkgSave.C
\end{enumerate}

\textbf{Background fits - NP}
\begin{enumerate}
\item ltBkg.C $\rightarrow$ calls plotLtPars.C
\begin{itemize}
\item ltPerPt.C $\rightarrow$ ltPerPt\_muFix.C $\rightarrow$ ltPerPt\_bFix.C
\end{itemize}
\end{enumerate}

\textbf{Background fits - mass bkg}
\begin{enumerate}
\item (bkgFits) newMCmass\_G.C
\item mBkg.C $\rightarrow$ calls plotDMPars.C
\item bkgCosth.C
\item fitBkgCosth2d.C $\rightarrow$ calls plotCosPars2d.C
\item getfL.C
\item genDist.C
\end{enumerate}

\textbf{Background functions}
\begin{enumerate}
\item fbkgProp.C
\item fnpProp.C
\item fNPcorr.C (this point onwards requires running the NP\_fit framework)
\end{enumerate}

\textbf{Final fit}
\begin{enumerate}
\item bkgSub.C
\item indFit.C
\item plotRes.C
\end{enumerate}

\pagebreak

\subsection{bkgFits}

\textbf{Base fits}
\begin{enumerate}
\item newMCmass\_0-5.C
\item newMCmass\_G.C
\end{enumerate}

\textbf{Base fit plots}
\begin{enumerate}
\item plotMMseq.C
\item plotnalpha.C
\item plotMMG.C
\end{enumerate}

\textbf{NP fits}
\begin{enumerate}
\item NPMCmass.C
\item NPMCmass\_hpt.C
\item NPMCmass\_hp\_fn.C
\item combNPRes.C
\end{enumerate}

\subsection{NP\_fit}

\textbf{Initial storage}
\begin{enumerate}
\item bkgSave.C
\end{enumerate}

\textbf{Background fits - mass bkg}
\begin{enumerate}
\item (bkgFits) NPMCmass.C
\item mBkg.C $\rightarrow$ calls plotDMPars.C
\item bkgCosth.C
\item fitBkgCosth2d.C $\rightarrow$ calls plotCosPars2d.C
\item getfL.C
\item genDist.C
\end{enumerate}

\textbf{Background functions}
\begin{enumerate}
\item fbkgProp.C
\end{enumerate}

\textbf{Final fit}
\begin{enumerate}
\item bkgSub.C
\item indFit.C
\item plotRes.C
\end{enumerate}

\pagebreak

\section{PR SR $\pt$ vs $\cost$}
\subsection{Regions} 

The three mass regions and the two lifetime regions are defined according to fixed limits
\begin{equation}\begin{cases}
m^{LSB}&\in[2.92, 2.95] \text{ GeV}\\
m^{SR}&\in[3.0, 3.2]  \text{ GeV}\\
m^{RSB}&\in[3.21, 3.28]  \text{ GeV}\\
\end{cases}\hspace{0.5cm}\text{and}\hspace{0.5cm}
\begin{cases}
|\ell^{PR}| < 0.005\text{ cm} \\
0.01 < \ell^{NP} < 0.05\text{ cm}
\end{cases}
\end{equation}

\subsection{Cuts and binning} 

Cuts on stored datafiles:
\begin{itemize}
\item $M(\mu\mu)\in[3.0,3.2]$ GeV
\item $|y(\mu\mu)|<1.2$
\item (PR) $|\ell|<0.005$ cm
\item (NP) $0.01 < \ell < 0.05$ cm
\end{itemize}

Binning for SR data (PR and NP) and MC (17 bins)
\begin{itemize}
\item $25-46$ GeV: 7 bins of 3 GeV
\item $46-76$ GeV: 6 bins of 5 GeV
\item $76-100$ GeV: 3 bins of 8 GeV
\item $100-120$ GeV: 1 bin of 20 GeV
\end{itemize}

\subsection{Storing histograms}

Run histoSave.C to store the 2d histos ($\pt$ vs $\cost$) in histoStore.root.

\subsection{Limit $\cost$}

In folder cosMax, run histoSave.C and getCos.C to get the limit $\cost$ in each $\pt$ and interpolate a function to get $\cost_{max}(\pt)$, using a finer binning for precision.
\begin{itemize}
\item $25-40$ GeV: 15 bins of 1 GeV
\item $40-100$ GeV: 30 bins of 2 GeV
\item $100-120$ GeV: 4 bin of 5 GeV
\end{itemize}

\pagebreak

\section{Background fitting: Lifetime and Mass distributions}

\subsection{Storing histograms}

Run bkgSave.C to store the 1d histos (mass in the prompt region and lifetime in the signal region) for all $\pt$ bins. Stores data, MC and data/MC ratio. The NP dimension uses the usual binning

Coarse binning for the mass dimension fitting (7 bins):
\begin{itemize}
\item $25-46$ GeV: 3 bins of 7 GeV
\item $46-76$ GeV: 3 bins of 10 GeV
\item $76-120$ GeV: 1 bin of 44 GeV
\end{itemize}

\subsection{Lifetime fit}

Fit the $\ell$ region $[-0.005,0.05]$ cm (mass in the signal region). For the PR contribution (equal to the resolution), consider a double Gaussian distribution:
\begin{equation}
L_{PR}=L_{res} = (1-f)\cdot g_{G_1}(\ell|\mu, \sigma_1)+f\cdot g_{G_2}(\ell|\mu, \sigma_2),
\end{equation}

Both Gaussian functions share a common $\mu$ and have independent $\sigma_{1,2}$. For the NP contribution, consider a negative exponential convoluted with the double Gaussian defined above:
\begin{equation}
L_{NP}=	\left[\exp(-\ell/\lambda)\cdot\mathcal{H}(\ell)\right]*L_{res}(\ell|\mu, \sigma_1,\sigma_2),
\end{equation}
where $\mathcal{H}$ is the Heaviside step function.

Running a 1d fit with free $\sigma_{1,2}$, $\lambda$, and normalizations $N_{PR}$, $N_{NP}$. We consider three sets of conditions:
\begin{itemize}
\item Free $\mu$, $f$
\item $\mu$ fixed to results of previous fit, free $f$
\item $\mu$ as above, $f$ fixed to results of previous fit
\end{itemize} 

\subsubsection{Running the fit}

Run ltBkg.C to run the 1d fits reading the histos from histoSave.C and outputting all fit parameters as well as $\chi^2/$ndf and NP fraction per $\pt$ bin, $f_{NP}$. Plots distribution, pulls and relative difference.

\pagebreak

\subsection{Mass fit}

Fit the mass region $[2.94, 3.26]$ GeV (lifetime in the prompt region). For each $\pt$ bin, fit the signal with a double crystal ball function and a Gaussian:
\begin{equation}
f_{CB1}\cdot g_{CB_1}(m_{\mu^+\mu^-})+(1-f_{CB1}-f_G)\cdot g_{CB_2}(m_{\mu^+\mu^-})+f_G\cdot g_G(m_{\mu^+\mu^-}),
\end{equation}
where a crystal ball function $g_{CB}(m_{\mu^+\mu^-})$ is defined by
\begin{equation}
g_{CB}(m) = \begin{cases}
\frac{N_{SR}}{\sqrt{2\pi}\sigma_{CB}}\exp\left(-\frac{(m-\mu)^2}{2\sigma^2_{CB}}\right), & \text{for }\frac{m-\mu}{\sigma_{CB}} > -\alpha \\
\frac{N_{SR}}{\sqrt{2\pi}\sigma_{CB}}\left(\frac{n}{|\alpha|}\right)^n\exp\left(-\frac{|\alpha|^2}{2}\right)\left(\frac{n}{|\alpha|}-|\alpha|-\frac{m-\mu}{\sigma_{CB}}\right)^{-n}, & \text{for }\frac{m-\mu}{\sigma_{CB}}\leq-\alpha
\end{cases}
\end{equation}
and a Gaussian $g_{G}(m_{\mu^+\mu^⁻})$ is defined by
\begin{equation}
g_{G}(m) = \frac{N_{SR}}{\sqrt{2\pi}\sigma_{G}}	\exp\left(-\frac{(m-\mu)^2}{2\sigma^2_{G}}\right).
\end{equation}
All functions share a common $N_{SR}$, $\mu$ and have independent $\sigma_{1,2,G}$. The CB share $n$ and $\alpha$. The background is modeled with a negative exponential
\begin{equation}
N_{BG} \text{e}^{- m / \lambda}
\end{equation}

Running a 2d fit with constant $f$ and $\mu$, linear $\sigma_{1,2}$ and free $\lambda$ and normalizations $N_{SR}$, $N_{BG}$. The tail parameters $n$ and $\alpha$ are fixed to results of the MC studies, as well as the Gaussian parameters $f_G$ (constant) and $\sigma_G$ (linear in $\pt$). 

\subsubsection{Running the fit}

Run mBkg.C to run the 2d fit reading the histos from bkgSave.C and outputting all fit parameters as well as $\chi^2/$ndf and mass background fraction per $\pt$ bin, $f_{bkg}$. Plots distribution, pulls and relative difference.

\pagebreak

\subsection{Plotting the fit parameters}

The lifetime fit parameters are plotted with plotLtPars.C. This plots:
\begin{itemize}
\item All 7 fit parameters
\item $f_{NP}$
\item $\sigma=\sqrt{\sigma_1^2+\sigma_2^2}$
\item rms$=\sigma\cdot\pt/M$
\item $\sigma_2/\sigma_2$
\end{itemize}

The mass fit parameters are plotted with plotDMPars.C. This plots:
\begin{itemize}
\item All 9 fit parameters
\item $f_{BG}$
\end{itemize}

\pagebreak

\section{Background fitting: $\cost$ distributions}
\subsection{Background selection}

LSB background selection:
\begin{itemize}
\item $M(\mu\mu)\in[2.92,2.95]$ GeV
\item $|y(\mu\mu)|<1.2$
\item $|c\tau| < 0.005$ cm
\end{itemize}

RSB background selection:
\begin{itemize}
\item $M(\mu\mu)\in[3.21,3.28]$ GeV
\item $|y(\mu\mu)|<1.2$
\item $|c\tau| < 0.005$ cm
\end{itemize}

\subsection{Storing histograms}

Run bkgHisto.C to store the 2d histos ($\pt$ vs $\cost$) of data/MC for the two mass background regions.

\subsection{$\cost$ fit}

For each $\pt$ bin, fit the distribution with
\begin{equation}
f_{\cost}^{SB} = N\left(1+\lambda_2(\cost)^2+\lambda_4(\cost)^4\right)
\end{equation}

\subsubsection{Sideband fit}

Running a 2d fit with normalization $N$ free and $\lambda_2$, $\lambda_4$ constant in $\pt$. Outputs all fit parameters per $\pt$ bin. Plots distributions, pulls and relative difference.

The sideband fit parameters are plotted with plotCosPars2d.C

\pagebreak

\section{Background subtraction}

\subsection{Getting the mass background distributions}

Generate the background $\cost$ distributions using the above fit parameters, according to the relative proportion given by $f_L$. This parameter is determined with getfL.C, and fixed to its central value (around 53.3\%). The distributions are obtained with their associated uncertainty, obtained by error propagation of the fit results.

The distributions are obtained in the 17 $\pt$ bins of the data. Done with genDist.C, with histos stored in bkgCosModel.root.

\subsection{The background fractions}

For each of the 7 (or 17) $\pt$ bins used in the associated fit, determine the resulting $f_{bkg}$ or $f_{NP}$ and propagate the error with the fit uncertainty and covariance of each input parameter. This is done with fbkgProp.C for $f_{bkg}$ and fNPProp.C for $f_{NP}$.

\subsubsection{Mass background fit}

Fit the mass background fraction, now with propagated uncertainties, with 
\begin{equation}
M\left(1-e^{-a\cdot(\pt-\mu)}\right)
\end{equation}
Also part of fbkgProp.C. Plots $f_{bkg}$ and stores the graph and the fit function, as well as the histogram with propagated uncertainties to be used in the background subtraction.

\subsubsection{NP fraction correction}

Run NPcorr.C to correct $f_{NP}$ for the mass bkg contamination of the NP distributions,
\begin{equation}
f_{NP}^c = f_{NP}\cdot\left(1-f_{bkg}^{NP}\right).
\end{equation}
Also plots the comparison $f_{NP}$ vs $f_{NP}^c$.

\pagebreak

\section{Final results}

\subsection{Running the subtraction}

Running code bkgSub.C runs the subtraction over the fine $\pt$ bins. Uses bkg/MC as determined through fits, NP/MC as determined by mass bkg correction of the NP distribution, fitted $f_{bkg}$ and $f_{NP}^c$. Stores
\begin{itemize}
\item 2d $\cost$ distributions of Peak/MC and scaled NP/MC, bkg/MC
\item 2d $\cost$ distributions of prompt/MC and prompt $\jpsi$/MC
\end{itemize}

\subsection{$\cost$ fit}

Code indFit.C runs the final fit, for  Data, NP, prompt and prompt J/$\psi$ $|\cost|$ distributions. Plots the distributions with superimposed fit on Peal, Prompt dimuon, and prompt J/$\psi$ and $\lambda_\theta$ results. Stores the fit $A$, $\lambda_\theta$ and $\chi^2$, $ndf$ for all cases. 

\subsection{Plotting results}

Plot $A$, $P(\chi^2)$, $\lambda_\theta$ of all and $\lambda_\theta$ of prompt J/$\psi$ with constant fit. Run code plotRes.C

\end{document}
