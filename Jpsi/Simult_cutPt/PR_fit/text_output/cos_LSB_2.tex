\begin{tabular}{c||c|c|c|c}
$\pt$ (GeV) & $N$ & $\lambda_{2}$ & $\lambda_4$  & $\chi^2$/ndf  \\
\hline
$[25, 32]$ & 0.0025216 $\pm$ 0.0000094 & 0.30 & 0.50 & 19/12\\
$[32, 39]$ & 0.002662 $\pm$ 0.000016 & 0.30 & 0.50 & 14/13\\
$[39, 46]$ & 0.002935 $\pm$ 0.000026 & 0.30 & 0.50 & 20/14\\
$[46, 56]$ & 0.001884 $\pm$ 0.000021 & 0.30 & 0.50 & 13/15\\
$[56, 66]$ & 0.001819 $\pm$ 0.000033 & 0.30 & 0.50 & 8/15\\
$[66, 76]$ & 0.000707 $\pm$ 0.000019 & 0.30 & 0.50 & 17/16\\
$[76, 120]$ & 0.000726 $\pm$ 0.000018 & 0.30 & 0.50 & 29/16\\
\end{tabular}
